\chapter{Mineração de Dados}

A Mineração de Dados é o processo de explorar e analisar grandes volumes de dados (geralmente de múltiplas fontes) para descobrir padrões, relacionamentos, tendências e informações úteis que seriam impossíveis de encontrar manualmente.

Pense nela como a "mineração" clássica: você tem uma montanha de terra e rocha (os dados brutos) e usa ferramentas e técnicas sofisticadas para extrair pepitas de ouro puro (o conhecimento valioso).

Ela está intrinsecamente ligada ao Big Data (que fornece o material bruto em grande volume, variedade e velocidade) e ao Aprendizado de Máquina (que fornece muitas das ferramentas automatizadas para encontrar os padrões).


\section{O Processo de Mineração de Dados}

O framework mais famoso e utilizado para guiar projetos de mineração de dados é o CRISP (Cross-Industry Standard Process for Data Mining). Ele é cíclico e iterativo.

\begin{enumerate}
	\item \textbf{Compreensão do Negócio}:
	
	\begin{itemize}
		\item Definir objetivos.
		\item Perguntar: Qual problema precisa ser resolvido?
	\end{itemize}

	\item \textbf{Compreensão dos Dados}:
	
	\begin{itemize}
		\item Coleta dos dados.
		\item Exploração inicial (estatísticas, gráficos, limpeza).
	\end{itemize}
	
	\item \textbf{Preparação dos Dados}:
	
	\begin{itemize}
		\item Limpeza (remover outliers, lidar com valores faltantes).
		\item Transformar (normalizar, redução de dimensionamento, feature engineering).
	\end{itemize}
	
	\item \textbf{Modelagem}:
	
	\begin{itemize}
		\item Aplicar algoritmos de classificação, regressão, clisterização, associação, etc.
	\end{itemize}
	
	\item \textbf{Avaliação}:
	
	\begin{itemize}
		\item Verificar métricas de desempenho (acurácia, precisão, recall, F1, RMSE, etc.).
		\item Validar se o modelo responde ao problema de negócio.
	\end{itemize}
	
	\item \textbf{Implantar}:
	
	\begin{itemize}
		\item Colocar o modelo em uso prático (ex: recomendador na Netflix).
	\end{itemize}
\end{enumerate}

\begin{figure}
	\centering
	\includegraphics[width=0.5\linewidth]{"Cursos/02 - Ciência de Dados, Aprendizado de Máquina e Mineração de Dados/images/ProcessoMineracaoDeDados"}
	\caption{}
	\label{fig:processomineracaodedados}
\end{figure}