\chapter{Agrupamento Particional}

O Agrupamento Particionado é um método de aprendizado não supervisionado que divide um conjunto de dados em um número pré-definido, $k$, de grupos mutuamente exclusivos, chamados de clusters. O objetivo é organizar os dados de modo que:

\begin{itemize}
	\item \textbf{Dentro de um cluster (Coesão)}: os objetos sejam o mais similares possível.
	
	\item \textbf{Entre clusters (Separação)}:  os objetos de clusters diferentes sejam o mais dissimilares possível.
\end{itemize}

A palavra "particionado" vem do fato de que cada ponto de dados pertence a exatamente um cluster (partição rígida).


\section{Para Que é Utilizado?}

O K-Means é extremamente versátil e aplicado em diversas áreas, graças a sua eficiência e simplicidade.

\begin{itemize}
	\item \textbf{Segmentação de Clientes}: Agrupar clientes com base em seu comportamento de compra, dados demográficos ou histórico de navegação para criar campanhas de marketing direcionadas.
	
	\item \textbf{Organização de Documentos}: Agrupar notícias ou artigos por tópicos similares.
	
	\item \textbf{Compressão de Imagens}: Reduzir o número de cores em uma umagem agrupando cores similares e substituindo-as pela cor do centroide do cluster.
	
	\item \textbf{Detecção de Anomalias}: Pontos de dados que ficam muito distantes de qualquer centroide após o agrupamento podem ser considerados anomalias ou outliers.
	
	\item \textbf{Análise de Dados Genômicos}: Agrupar genes com padrões de expressão semelhantes.
\end{itemize}

\section{O Algoritmo K-Means}

O K-Means é o algoritmo mais famoso de agrupamento particionado. Sua ideia é simples: encontrar $k$ centróides (o "centro" de cada cluster) e atribuir cada ponto ao centróide mais próximo, minimizando a variância dentro dos clusters.

\textbf{Fórmula}

O algoritmo busca minimizar a Soma dos Erros Quadráticos (SSE - Sum of Squared Errors), também conhecida como inércia dentro do cluster.

\begin{center}
	$E = \sum_{i=1}^{k}\sum_{x \in C_i}^{} ||x, \mu_i||^2$
\end{center}

\begin{itemize}
	\item $k$: Número de clusters.
	
	\item $C_i$: Conjunto de pontos no cluster $i$.
	
	\item $\mu_i$: Centroid do cluster $i$.
	
	\item $||x, \mu_i||^2$: Distância Euclidiana ao quadrado entre o ponto $x$ e o seu centroide.
\end{itemize}

\textbf{Passos do Algoritmo (Explicado com a Figura Acima)}

\begin{enumerate}
	\item \textbf{Inicialização}: Escolha aleatória de $k$ centróides iniciais (\ref{fig:k-means-inicializacao}).
	
	\begin{figure}
		\centering
		\includegraphics[width=0.7\linewidth]{"Cursos/02 - Ciência de Dados, Aprendizado de Máquina e Mineração de Dados/images/K-Means-Inicializacao"}
		\caption{Inicialização}
		\label{fig:k-means-inicializacao}
	\end{figure}
	
	\item \textbf{Atribuição de Pontos}: Cada ponto é atribuído ao centróide mais próximo, formando $k$ clusters temporários (\ref{fig:k-means-atribuicao}).
	
	\begin{figure}
		\centering
		\includegraphics[width=0.7\linewidth]{"Cursos/02 - Ciência de Dados, Aprendizado de Máquina e Mineração de Dados/images/K-Means-Atribuicao"}
		\caption{Atribuição}
		\label{fig:k-means-atribuicao}
	\end{figure}
	
	\item  \textbf{Atualização dos Centroides}: O centróide de cada cluster é recalculado como a média de todos os pontos naquele cluster (\ref{fig:k-means-atualizacao}).
	
	\begin{figure}
		\centering
		\includegraphics[width=0.7\linewidth]{"Cursos/02 - Ciência de Dados, Aprendizado de Máquina e Mineração de Dados/images/K-Means-Atualizacao"}
		\caption{Atualização}
		\label{fig:k-means-atualizacao}
	\end{figure}
	
	\item \textbf{Iteração}: Os passos 2 e 3 são repetidos até que não haja mais mudanças na atribuição dos pontos ou um número máximo de iterações seja atingido (convergência para o Resultado Final).
\end{enumerate}


\section{Exemplos}

\subsection{Exemplo Descritivo: Segmentação de Clientes de E-commerce}

Imagine um e-commerce que quer segmentar seus clientes com base em duas variáveis: "Frequência de Visitas" e "Valor Gasto".

\begin{enumerate}
	\item \textbf{Escolha do k}: A equipe de marketing decide que 3 segmentos (ex: "ocasionais", "fiéis" e "super fãs") é um bom ponto de partida.

	\item \textbf{Inicialização}: O algoritmo K-Means escolhe 3 clientes aleatórios como os centroides iniciais.
	
	\item \textbf{Execução}:
	
	\begin{itemize}
			\item Na Primeira iteração, cada cliente é atribuído ao centróide mais próximo.
			
			\item Nas próximas iterações, os clientes são realocados conforme os centróides se movem.
			
			\item O processo converge quando os centróides se estabilizam e os clusters estão bem definidos.
	\end{itemize}
		
	\item \textbf{Resultado}:
	
	\begin{itemize}
		\item \textbf{Cluster 1 (Fiéis)}: Alta frequência, alto valor gasto.
		
		\item \textbf{Cluster 2 (Ocasionais)}: Média frequência, médio valor gasto.
		
		\item \textbf{Cluster 3 (Em Risco)}: Baixa frequência, baixo valor gasto (ou que não compram há tempos).
	\end{itemize}
\end{enumerate}

A empresa pode agora criar campanhas de marketing específicas para cada grupo.

\subsection{Exemplo Prático em Python}

Vamos usar scikit-learn para aplicar o K-Means em dados sintéticos com 4 clusters bem definidos.

\begin{lstlisting}[language=Python]
# Importando bibliotecas necessarias
from sklearn.cluster import KMeans
from sklearn.datasets import make_blobs
import matplotlib.pyplot as plt
import numpy as np

# 1. Criar um conjunto de dados sintetico para demonstracao
X, y_true = make_blobs(n_samples=300, centers=4, cluster_std=0.60, random_state=0)

# 2. Visualizar os dados originais
plt.scatter(X[:, 0], X[:, 1], s=50)
plt.title("Dados Originais")
plt.show()

# 3. Instanciar e executar o algoritmo K-Means com k=4
kmeans = KMeans(n_clusters=4, n_init=10, random_state=0)
kmeans.fit(X)
y_kmeans = kmeans.predict(X)

# 4. Visualizar os resultados do clustering
# Pontos coloridos por cluster
plt.scatter(X[:, 0], X[:, 1], c=y_kmeans, s=50, cmap='viridis')
# Plotar os centroides finais
centers = kmeans.cluster_centers_
plt.scatter(centers[:, 0], centers[:, 1], c='red', s=200, alpha=0.75, marker='X')
plt.title("Resultado do Clustering K-Means")
plt.show()

# 5. Mostrar as coordenadas dos centroides e a inertia (SSE)
print("Coordenadas dos Centroides:")
print(centers)
print(f"\nInércia (SSE): {kmeans.inertia_:.2f}")
\end{lstlisting}

\textbf{Saída Experada}:

\begin{itemize}
	\item Um primeiro gráfico com pontos azuis espalhados (\ref{fig:k-means-grafico-1}).
	
	\begin{figure}
		\centering
		\includegraphics[width=0.7\linewidth]{"Cursos/02 - Ciência de Dados, Aprendizado de Máquina e Mineração de Dados/images/K-Means-Grafico-1.png"}
		\caption{Atualização}
		\label{fig:k-means-grafico-1}
	\end{figure}
	
	\item segundo gráfico com os pontos coloridos em 4 grupos diferentes e um "X" vermelho marcando o centro de cada grupo (\ref{fig:k-means-grafico-2}).
	
	\begin{figure}
		\centering
		\includegraphics[width=0.7\linewidth]{"Cursos/02 - Ciência de Dados, Aprendizado de Máquina e Mineração de Dados/images/K-Means-Grafico-2.png"}
		\caption{Atualização}
		\label{fig:k-means-grafico-2}
	\end{figure}
	
	\item As coordenadas dos centróides e o valor da SSE serão impressos no console.
\end{itemize}

\section{Pontos Positivos e Negativos}

\subsection{Pontos Positivos}

\begin{itemize}
	\item \textbf{Simplicidade e Interpretabilidade}: O algoritmo é fácil de entender e implementar, e os resultados (clusters definidos por centroides) são fáceis de interpretar.
	
	\item \textbf{Eficiência Computacional}: O K-Means é relativamente rápido e escala bem para grandes conjunto de dados, o que torna desejável para Mineração de Dados.
	
	\item \textbf{Versatilidade}: Adapta-se bem a uma grande variedade de problemas e domínios.
	
	\item \textbf{Convergência Rápida}: Geralmente converge em poucas iterações.
\end{itemize}

\subsection{Pontos Negativos}

\begin{itemize}
	\item  \textbf{Necessidade de Definir $k$}: É preciso saber o número de cluster de antemão, o que nem sempre é óbvio.
	
	\item  \textbf{Sensibilidade a Inicialização}: A inicialização aleatória dos centróides pode levar a resultados diferentes (clusters sub-ótimos) a cada execução. A solução comum é rodar o algoritmo várias vezes com diferentes inicializações e escolher o melhor resultado.
		
	\item  \textbf{Sensibilidade a Outliers}: Como os centroides são baseados na média, eles são fortemente influenciados por outliers.
			
	\item  \textbf{Assume Clusters Esféricos}: O K-Means funciona melhor quando os clusters são de formato globular (esférico), de tamanhos e densidades similares. Ele tem dificuldade em encontrar clusters com formatos irregulares ou alongados.
\end{itemize}


\section{Conclusão}

O K-Means é um algoritmo fundamental no toolkit de qualquer cientista de dados. Sua simplicidade e eficiência o tornam uma excelente primeira escolha para tarefas de clustering, especialmente com grandes volumes de dados (Bid Data), e uma base sólida para comparação com algoritmos mais complexos.

No entanto, é crucial estar ciente de suas limitações. Para problemas com clusters de formato não-globular, algoritmos baseados em densidade como o DBSCAN ou baseados em hieraquia como Agrupamento Hierárquico são geralmente escolhas mais adequadas.