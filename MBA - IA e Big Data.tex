
\documentclass[
% -- opções da classe memoir --
12pt,		% tamanho da fonte
openright,	% capítulos começam em pág ímpar (insere página vazia caso preciso)
twoside,  % para impressão em anverso (frente) e verso. Oposto a oneside - Nota: utilizar \imprimirfolhaderosto*
%oneside, % para impressão em páginas separadas (somente anverso) -  Nota: utilizar \imprimirfolhaderosto
% inclua uma % antes do comando twoside e exclua a % antes do oneside 
a4paper,			% tamanho do papel. 
% -- opções da classe abntex2 --
chapter=TITLE,		% títulos de capítulos convertidos em letras maiúsculas
% -- opções do pacote babel --
english,			% idioma adicional para hifenização
french,				% idioma adicional para hifenização
spanish,			% idioma adicional para hifenização
brazil				% o último idioma é o principal do documento
% {USPSC-classe/USPSC} configura o cabeçalho contendo apenas o número da página
]{USPSC-classe/USPSC}
%]{USPSC-classe/USPSC1}
% Inclua % antes de ]{USPSC-classe/USPSC} e retire a % antes de %]{USPSC-classe/USPSC1} para utilizar o 
% cabeçalho diferenciado para as páginas pares e ímpares:
%- páginas ímpares: com seções ou subseções e o número da página
%- páginas pares: com o número da página e o título do capítulo 
% ---
% ---
% Pacotes básicos - Fundamentais 
% ---
\usepackage[T1]{fontenc}		% Seleção de códigos de fonte.
\usepackage[utf8]{inputenc}		% Codificação do documento (conversão automática dos acentos)
\usepackage{lmodern}			% Usa a fonte Latin Modern
\usepackage{kerkis}
% Para utilizar a fonte Times New Roman, inclua uma % no início do comando acima  "\usepackage{lmodern}"
% Abaixo, tire a % antes do comando  \usepackage{times}
%\usepackage{times}		    	% Usa a fonte Times New Roman	
% Para usar a fonte , lembre-se de tirar a % do comando %\renewcommand{\ABNTEXchapterfont}{\rmfamily}, localizado mais abaixo, logo após "Outras opções para nota de rodapé no Sistema Numérico" 					
\usepackage{lastpage}			% Usado pela Ficha catalográfica
\usepackage{indentfirst}		% Indenta o primeiro parágrafo de cada seção.
\usepackage{color}				% Controle das cores
\usepackage{graphicx}			% Inclusão de gráficos
\usepackage{float} 				% Fixa tabelas e figuras no local exato
\usepackage{chemfig}            % Para escrever reações químicas
\usepackage{chemmacros}         % Para escrever reações químicas
\usepackage{tikz}				% Para escrever reações químicas e outros
\usetikzlibrary{positioning}
\usepackage{microtype} 			% para melhorias de justificação
\usepackage{pdfpages}
\usepackage{makeidx}            % para gerar índice remissivo
\usepackage{hyphenat}         		 % Pacote para retirar a hifenizacao DO TEXTO
\usepackage[absolute]{textpos}	% Pacote permite o posicionamento do texto
\usepackage{eso-pic}          			% Pacote para incluir imagem de fundo
\usepackage{makebox}           		 % Pacote para criar caixa de 
\usepackage{listings} 					 %Pacote para criar code snuppet.
\usepackage{listingsutf8}
\usepackage{xcolor}
\usepackage[utf8]{inputenc}

%New colors defined below
\definecolor{codegreen}{rgb}{0,0.6,0}
\definecolor{codegray}{rgb}{0.5,0.5,0.5}
\definecolor{codepurple}{rgb}{0.58,0,0.82}
\definecolor{backcolour}{rgb}{0.95,0.95,0.92}

%Code listing style named "mystyle"
\lstdefinestyle{mystyle}{
	backgroundcolor=\color{backcolour},
	commentstyle=\color{codegreen},
	keywordstyle=\color{magenta},
	numberstyle=\tiny\color{codegray},
	stringstyle=\color{codepurple},
	basicstyle=\ttfamily\footnotesize,
	breakatwhitespace=false,         
	breaklines=true,                 
	captionpos=b,                    
	keepspaces=true,                 
	numbers=left,                    
	numbersep=5pt,                  
	showspaces=false,                
	showstringspaces=false,
	showtabs=false,                  
	tabsize=2,
	extendedchars=true,
	inputencoding=utf8/latin1,
}

%"mystyle" code listing set
\lstset{style=mystyle}
% ---

% ---
% Pacotes de citações
% Citações padrão ABNT
% ---
% Sistemas de chamada: autor-data ou numérico.
% Sistema autor-data
\usepackage[alf, abnt-emphasize=bf, abnt-thesis-year=both, abnt-repeated-author-omit=no, abnt-last-names=abnt, abnt-etal-cite=3, abnt-etal-list=3, abnt-etal-text=it, abnt-and-type=e, abnt-doi=doi, abnt-url-package=none, abnt-verbatim-entry=no]{abntex2cite}
\bibliographystyle{USPSC-classe/abntex2-alf-USPSC}
% Se o idioma for o inglês, inclua % no comando acima e exclua o % do comando abaixo
%\bibliographystyle{USPSC-classe/abntex2-alfeng-USPSC}

% Para o IQSC, que indica todos os autores nas referências, incluir % no início dos comandos acima e retirar a % dos comandos abaixo 
%\usepackage[alf, abnt-emphasize=bf, abnt-thesis-year=both, abnt-repeated-author-omit=no, abnt-last-names=abnt, abnt-etal-cite=3, abnt-etal-list=0, abnt-etal-text=it, abnt-and-type=e, abnt-doi=doi, abnt-url-package=none, abnt-verbatim-entry=no]{abntex2cite} 
%\bibliographystyle{USPSC-classe/abntex2-alf-USPSC}
% Se o idioma for o inglês, exclua % no comando acima ou do comando abaixo
%\bibliographystyle{USPSC-classe/abntex2-alfeng-USPSC}

% Sistema Numérico
% Para citações numéricas, sistema adotado pelo IFSC, incluir % no início dos comandos acima e retirar a % dos comandos abaixo 
%\usepackage{cite}              % agrupa citações numéricas consecutivas
%\usepackage[num, abnt-emphasize=bf, abnt-thesis-year=both, abnt-repeated-author-omit=no, abnt-last-names=abnt, abnt-etal-cite=3, abnt-etal-list=3, abnt-etal-text=it, abnt-and-type=e, abnt-doi=doi, abnt-url-package=none, abnt-verbatim-entry=no]{abntex2cite} 
%\bibliographystyle{USPSC-classe/abntex2-num-USPSC}
% Se o idioma for o inglês, exclua % no comando acima ou do comando abaixo
%\bibliographystyle{USPSC-classe/abntex2-numeng-USPSC}

% Complementarmente, verifique as instruções abaixo sobre os Pacotes de Nota de rodapé
% ---
% Pacotes de Nota de rodapé
% Configurações de nota de rodapé

% O presente modelo adota o formato numérico para as notas de rodapés quando utiliza o sistema de chamada autor-data para citações e referências. Para utilizar o sistema de chamada numérico para citações e referências, habilitar um dos comandos abaixo.
% Há diversa opções para nota de rodapé no Sistema Numérico.  Para o IFSC, habilitade o comando abaixo.

%\renewcommand{\thefootnote}{\fnsymbol{footnote}}  %Comando para inserção de símbolos em nota de rodapé

% Outras opções para nota de rodapé no Sistema Numérico:
%\renewcommand{\thefootnote}{\alph{footnote}}      %Comando para inserção de letras minúscula em nota de rodapé
%\renewcommand{\thefootnote}{\Alph{footnote}}      %Comando para inserção de letras maiúscula em nota de rodapé
%\renewcommand{\thefootnote}{\roman{footnote}}     %Comando para inserção de números romanos minúsculos  em nota de rodapé
%\renewcommand{\thefootnote}{\Roman{footnote}}     %Comando para inserção de números romanos minúsculos  em nota de rodapé

\renewcommand{\footnotesize}{\small} %Comando para diminuir a fonte das notas de rodapé
%Para utilizar a fonte Times New Roman, inclua retire % do início do comando abaixo 
%\renewcommand{\ABNTEXchapterfont}{\rmfamily}

% ---
% Pacote para agrupar a citação numérica consecutiva
% Quando for adotado o Sistema Numérico, a exemplo do IFSC, habilite 
% o pacote cite abaixo retirando a porcentagem antes do comando abaixo
%\usepackage[superscript]{cite}	

% ---
% Pacotes adicionais, usados apenas no âmbito do Modelo Canônico do abnteX2
% ---
\usepackage{lipsum}				% para geração de dummy text
% ---

% pacotes de tabelas
\usepackage{multicol}	% Suporte a mesclagens em colunas
\usepackage{multirow}	% Suporte a mesclagens em linhas
\usepackage{longtable}	% Tabelas com várias páginas
\usepackage{threeparttablex}    % notas no longtable
\usepackage{array}

% ----
% Compatibilização com a ABNT NBR 6023:2018 e 10520:2023
% Para tirar <> da URL e tornar as expressões latinas em itálico
\usepackage{USPSC-classe/ABNT6023-10520}
% As demais compatibilizações estão nos arquivos abntex2-alf-USPSC.bst,abntex2-alfeng-USPSC.bst, abntex2-num-USPSC.bst e abntex2-numeng-USPSC.bst, dependendo do idioma do textos e se o sistemas de chamada for autor-data ou numérico, conforme explicitado acima.
% ----

% ---
% DADOS INICIAIS - Define sigla com título, área de concentração e opção do Programa 
% Consulte a tabela referente aos Programas, áreas e opções de sua unidade contante do
% arquivo USPSC-Siglas estabelecidas para os Programas de Pós-Graduação nos APÊNDICES B-J
\siglaunidade{ICMC-TCC}
\programa{BCCp}
% Os demais dados deverão ser fornecidos no arquivo USPSC-pre-textual-UUUU ou USPSC-TCC-pre-textual-UUUU, onde UUUU é a sigla da Unidade. 
% Exemplo:USPSC-pre-textual-IFSC.tex
% ---
% Configurações de aparência do PDF final
% alterando o aspecto da cor azul
\definecolor{blue}{RGB}{41,5,195}

% informações do PDF
\makeatletter
\hypersetup{
	%pagebackref=true,
	pdftitle={\@title}, 
	pdfauthor={\@author},
	pdfsubject={\imprimirpreambulo},
	pdfcreator={LaTeX with abnTeX2},
	pdfkeywords={abnt}{latex}{abntex}{USPSC}{trabalho acadêmico}, 
	colorlinks=true,       		% false: boxed links; true: colored links
	linkcolor=black,          	% color of internal links
	citecolor=black,        		% color of links to bibliography
	filecolor=black,      		% color of file links
	urlcolor=black,
	%Para habilitar as cores dos links, retire a % antes dos comandos abaixo e inclua a % antes das 4 linhas de comando acima 
	%linkcolor=blue,            	% color of internal links
	%citecolor=blue,        		% color of links to bibliography
	%filecolor=magenta,      		% color of file links
	%urlcolor=blue,
	bookmarksdepth=4	
}
\makeatother
% --- 
% --- 
% Espaçamentos entre linhas e parágrafos 
% --- 

% O tamanho do parágrafo é dado por:
\setlength{\parindent}{1.3cm}

% Controle do espaçamento entre um parágrafo e outro:
\setlength{\parskip}{0.2cm}  % tente também \onelineskip

% ---
% compila o sumário e índice
\makeindex
% ---

% ----
% Início do documento
% ----
\begin{document}

% Seleciona o idioma do documento (conforme pacotes do babel)
\selectlanguage{brazil}
% Se o idioma do texto for inglês, inclua uma % antes do 
%      comando \selectlanguage{brazil} e 
%      retire a % antes do comando abaixo
%\selectlanguage{english}

% Retira espaço extra obsoleto entre as frases.
\frenchspacing 

% --- Formatação dos Títulos
\renewcommand{\ABNTEXchapterfontsize}{\fontsize{12}{12}\bfseries}
\renewcommand{\ABNTEXsectionfontsize}{\fontsize{12}{12}\bfseries}
\renewcommand{\ABNTEXsubsectionfontsize}{\fontsize{12}{12}\normalfont}
\renewcommand{\ABNTEXsubsubsectionfontsize}{\fontsize{12}{12}\normalfont}
\renewcommand{\ABNTEXsubsubsubsectionfontsize}{\fontsize{12}{12}\normalfont}

% ----------------------------------------------------------
% ELEMENTOS PRÉ-TEXTUAIS
% ----------------------------------------------------------
% ---
% Capa
% ---
%imprimircapa
% 26/03/2021 ----------------
%Capa do ICMC
\include{USPSC-TA-PreTextual/USPSC-CapaICMC}
\AddToShipoutPicture{\BackgroundBranco}

% ---
% inserir o sumario
% ---
\pdfbookmark[0]{\contentsname}{toc}
\tableofcontents*
\cleardoublepage
% ---


% ----------------------------------------------------------
% ELEMENTOS TEXTUAIS
% ----------------------------------------------------------
\textual

\part[Curso 2 - Ciência de Dados, Aprendizado de Máquina e Mineração de Dados]{Curso 2 - Ciência de Dados, Aprendizado de Máquina e Mineração de Dados}
% ---
\include{Cursos/02 - Ciência de Dados, Aprendizado de Máquina e Mineração de Dados/NaiveBayes.tex}

\include{Cursos/02 - Ciência de Dados, Aprendizado de Máquina e Mineração de Dados/ArvoreDeDecisao.tex}

\include{Cursos/02 - Ciência de Dados, Aprendizado de Máquina e Mineração de Dados/AvaliacaoDeClassificadores.tex}

\include{Cursos/02 - Ciência de Dados, Aprendizado de Máquina e Mineração de Dados/RandomForest.tex}

\include{Cursos/02 - Ciência de Dados, Aprendizado de Máquina e Mineração de Dados/SupportVectorMachine.tex}

\chapter[K-Nearest Neighbors - KNN]{K-Nearest Neighbors - KNN}

O KNN é um algoritomo de aprendizado supervisionado usado para classificação e regressão. Ele é um algoritmo não paramétrico e baseado em instâncias (ou lazy learning).

A ideia central por trás do KNN é extremamente intuitiva e segue um princípio fundamental: "Diga-me com quem andas e eu te direi quem és".

\begin{itemize}
	\item \textbf{Não paramétrico}: Significa que o algoritmo não faz nenhuma suposição sobre a distribuição subjacente dos dados. Ele é livre para aprender qualquer formato de função de decisão.
	
	\item \textbf{Baseado em instâncias / Laze Learning}: Significa que o algoritmo não aprende um modelo explícito durante a fase de treinamento. Em vez disso, ele simplesmente "memoriza" (armazena) todo o conjunto de dados de treinamento. Todo o cálculo pesado é adiado para a fase de previsão.
\end{itemize}


\section{Para que é utilizado?}

O KNN é um algoritmo versátil aplicado em diversos contextos:

\begin{enumerate}
	\item \textbf{Reconhecimento de Padrões}:
	
	\begin{itemize}
		\item \textbf{Classificação de Texto}: Categorizar documentos.
		
		\item \textbf{Reconhecimento de Imagem}: Reconhecimento de dígitos escritos à mão ou de gestos.
		
		\item \textbf{Sistemas de Recomendação}: Recomendar produtos ou conteúdos com base no que usuários similares gostaram (é a base dos filtros colaborativos). Exemplo: "Quem comprou este produto também comprou...".
	\end{itemize}
	
	\item \textbf{Previsões na Medicina}:
	
	\begin{itemize}
		\item Prever o risco de uma doença com base em pacientes com características similares (idade, simtomas, resultados de exames).
	\end{itemize}
	
	\item \textbf{Problemas de Regressão}:

	\begin{itemize}
		\item Prever o valor de uma casa com base no preço de venda de propriedades similares no mesmo bairro.
	\end{itemize}	
\end{enumerate}


\section{O Algoritmo: Como Funciona? (Conceitos e Passos)}

O funcionamento do KNN pode ser resumido em três passos simples para um novo ponto de dados.

\begin{enumerate}
	\item \textbf{Calcular a Distância}: Calcular a distância entre o novo ponto (a ser classificado/previsto) e todos os outros pontos no conjunto de dados de treinamento armazenado.
	
	\item \textbf{Encontrar os K-Vizinhos}: Identificar os K pontos no conjunto de treinamento que estão mais próximos do novo ponto (os K vizinhos mais próximos).
	
	\item \textbf{Realizar a Votação (Classificação) ou Média (Regressão)}:
	
	\begin{itemize}
		\item \textbf{Classificação}: A classe do novo ponto é determinada pela classe majoritária entre seus K vizinhos.
		
		\item \textbf{Regressão}: O valor do novo ponto é a média (ou a média ponderada) dos valores dos seus K vizinhos.
	\end{itemize}
\end{enumerate}


\section{A Fórmula Chave: Medida de Distância}

O coração do algoritmo KNN é a função de distância. Ela define o que significa "próximo" ou "similar".


\subsection{Distância Euclidiana (a mais comum)}

A distância em linha reta entre dois pontos no espaço euclidiano.

\textbf{Fórmula 2-D}:

$d(p, q) = \sqrt{(q_x - p_x)^2 + (q_y - p_y)^2}$

\textbf{Fórmula n-Dimensional}:

$d(p, q) = \sqrt{\sum_{i=1}^{n} (q_i - p_i)^2}$


\subsection{Distância de Manhattan (Distância do Quarteirão)}

A soma das diferenças absolutas das coordenadas. É menos sensível a outliers que a Euclidiana.

\textbf{Fórmula}:

$d(p, q) = \sum_{i=1}^{n} |q_i - p_i|$


\subsection{Distância de Minkowski}

Uma generalização das distâncias Euclidiana e Manhattan.

\textbf{Fórmula}:

$d(p, q) = \left( \sum_{i=1}^{n} |q_i - p_i|^p \right)^{1/p}$

\begin{itemize}
	\item Se $p=1$, é a Distância de Manhattan.
	
	\item Se $p=2$, é a Distância Euclidiana.
\end{itemize}


\subsection{Distância de Hamming}

Usada para dadas categóricos. É simplesmente a proporção de características que diferem.

Exemplo: Se $p = [1, 0, 1, 1] $ e $q = [1, 1, 1, 0]$, a distância de Hamming é 2 (as características nas posições 2 e 4 são diferentes).


\section{Exemplo Prático: Classificação de um Novo Paciente}

Suponha um dataset onde classificamos se um paciente tem uma condição cardíaca (Sim ou Não) com base em dois features: Idade e Nível de Colesterol.

\textbf{Conjunto de Treinamento}:

\begin{table}[ht]
	\centering
	\begin{tabular}{cccc}
		\hline
		\textbf{Paciente} & \textbf{Idade} & \textbf{Colesterol} & \textbf{Condição Cardíaca} \\ \hline
		A                 & 50             & 200                 & Não                        \\
		B                 & 60             & 230                 & Sim                        \\
		C                 & 55             & 190                 & Não                        \\
		D                 & 65             & 250                 & Sim                        \\
		\hline
	\end{tabular}
\end{table}

\textbf{Novo Paciente}: Idade = 58, Colesterol = 210. Ele tem a condição?

\begin{enumerate}
	\item \textbf{Calcular Distâncias (usando Euclidiana)}:
	
	\begin{itemize}
		\item Distância até A: $\sqrt{(58-50)^2 + (210-200)^2} = \sqrt{64 + 100} = \sqrt{164} \approx 12.8$.
		
		\item Distância até B: $\sqrt{(58-60)^2 + (210-230)^2} = \sqrt{4 + 400} = \sqrt{404} \approx 20.1$.
		
		\item Distância até C: $\sqrt{(58-55)^2 + (210-190)^2} = \sqrt{9 + 400} = \sqrt{409} \approx 20.2$.
		
		\item Distância até D: $\sqrt{(58-65)^2 + (210-250)^2} = \sqrt{49 + 1600} = \sqrt{1649} \approx 40.6$.
	\end{itemize}
	
	\item \textbf{Encontrar os K vizinhos mais próximos (Vamos escolher K=3)}:
	
	\begin{itemize}
		\item 1º vizinho mais próximo: A (dist=12.8).
		
		\item 2º vizinho mais próximo: B (dist=20.1).
		
		\item 3º vizinho mais próximo: C (dist=20.2).
	\end{itemize}
	
	\item \textbf{Votação}:
	
	\begin{itemize}
		\item Os 3 vizinhos são: [A: "Não", B: "Sim", C: "Não"].
		
		\item A classe majoritária é "Não" (2 votos contra 1).
	\end{itemize}
\end{enumerate}

\textbf{Previsão}: O novo paciente "Não" tem a condição cardíaca.

\section{Exemplo em Python}

\begin{lstlisting}[language=Python]
# Importando bibliotecas
import matplotlib.pyplot as plt
from sklearn.datasets import make_classification
from sklearn.model_selection import train_test_split
from sklearn.neighbors import KNeighborsClassifier
from sklearn.metrics import classification_report, confusion_matrix
import numpy as np

# 1. Criar dataset sintetico (2D para visualizacao)
X, y = make_classification(
n_samples=200,     # 200 pontos
n_features=2,      # 2 dimensoes (x,y)
n_classes=2,       # binario
n_clusters_per_class=1,
random_state=42
)

# 2. Dividir em treino e teste
X_train, X_test, y_train, y_test = train_test_split(X, y, test_size=0.3, random_state=42)

# 3. Treinar modelos com diferentes valores de K
ks = [1, 5, 15]
plt.figure(figsize=(15,4))

for i, k in enumerate(ks):
clf = KNeighborsClassifier(n_neighbors=k)
clf.fit(X_train, y_train)

# 4. Avaliacao
y_pred = clf.predict(X_test)
print(f"\n=== Resultados para K={k} ===")
print(confusion_matrix(y_test, y_pred))
print(classification_report(y_test, y_pred))

# 5. Plot da fronteira de decisao
h = .02  # passo da malha
x_min, x_max = X[:, 0].min() - 1, X[:, 0].max() + 1
y_min, y_max = X[:, 1].min() - 1, X[:, 1].max() + 1
xx, yy = np.meshgrid(np.arange(x_min, x_max, h),
np.arange(y_min, y_max, h))

Z = clf.predict(np.c_[xx.ravel(), yy.ravel()])
Z = Z.reshape(xx.shape)

plt.subplot(1, len(ks), i+1)
plt.contourf(xx, yy, Z, alpha=0.3)
plt.scatter(X_train[:, 0], X_train[:, 1], c=y_train, marker="o", label="treino")
plt.scatter(X_test[:, 0], X_test[:, 1], c=y_test, marker="x", label="teste")
plt.title(f"K = {k}")
plt.legend()

plt.tight_layout()
plt.show()
\end{lstlisting}

\textbf{O que este código faz}:

\begin{enumerate}
	\item Gera um dataset sintético 2D para visualização.
	
	\item Separa em treino (70\%) e teste (30\%).
	
	\item Treina o KNN com K=2, 5 e 15.
	
	\item Mostra os relatórios de classificação.
	
	\item Plota as fronteiras de decisão para ver como o K muda o comportamento.
\end{enumerate}


\section{Pontos Positivos (Vantagens)}

\begin{itemize}
	\item \textbf{Simplicidade e Intuição}: Extremamente fácil de entender e implementar. É um excelente algoritmo para criar uma baseline.
	
	\item \textbf{Não Há Fase de Treinamento}: Como é um lazy learner, o "treinamento" é instantâneo (apenas amrmazenar os dados). Isso é vantajoso se o conjunto de dados mudar frequentemente, ou seja, novos dados podem ser adicionados sem necessidade de retreinar um modelo complexo.
	
	\item \textbf{Adaptativo}: O algoritmo se adapta naturamente à medida que novos dados de treinamento são coletado.
	
	\item \textbf{Versátil}: Funciona para classificação e regressão.
\end{itemize}


\section{Pontos Negativos (Desvantagens) - Críticos no Contexto de Big Data}

\begin{itemize}
	\item \textbf{Computacionalmente, Extremamente Ineficiente na Previsão}: Esta é a principal desvantagem. Para prever um único ponto, o algoritmo precisa calcular a distância para todos os pontos no conjunto de treinamento. A complexidade para uma única previsão é $O(n*d)$, onde $n$ é o número de amostras e $d$ o número de features. Para grande volumes de dados (Big Data), isso se torna proibitivamente lento e inviável para aplicações em tempo real.
	
	\item \textbf{Sensibilidade à Escala das Features}: Se uma feature tem uma escala muito maior que outr (ex: salário vs idade), ele dominará completamente o cálculo da distância. É crucial normalizar ou padronizar os dados antes de usar o KNN.
	
	\item \textbf{Maldição da Dimensionalidade}: O desempenho do KNN degrada severamente à medida que o número de features ($d$) aumenta. em espaços de muito alta dimensionalidade, o conceito de "proximidade" perde o significado, pois todos os pontos tendem a estar equidistantes uns dos outros.
	
	\item \textbf{Sensibilidade à Escoilha de $K$ e da Métrica de Distância}: A escolha di hiperparâmetro $K$ é crucial.
	
	\begin{itemize}
		\item \textbf{K muito pequeno (ex: K=1)}: Modelo muito complexo, sujeito a overfitting e ruído. A fronteira de decisão é muito irregular.
		
		\item \textbf{K muit grade (ex: K=100)}: Modelo muito simples, sujeito a underfitting. A fronteira de decisão é muito suave e pode ignorar padrões importantes.
	\end{itemize}
	
	\item \textbf{Requer Armazenamento do Conjunto de Dados Inteiro}: Em sistemas com restrições de memória, armazenar gigabytes ou terabytes de dados de treinamento apenas para fazer previsões pode ser impossível.
\end{itemize}


\section{Conclusão}

O KNN é um algoritmo conceitualmente simples e poderoso para problemas de pequena escala e baixa dimensionalidade. Sua intuitividad o torna uma ferramenta pedagógica valiosa e uma boa primeira tentativa para problemas simples.

No entanto, no contexto de Big Data e Aprendizado de Máquina aplicado, suas desvantagens - especialmente seu custo computacional proibitivo na fase de previsão - o tornam uma escolha pouco prática. Ele é raramente usado em produção para lidar com milhões ou bilhões de amostras.

Algoritmos como Árvores de Decisão, Random Forest, XGBoost e até mesmo SVM (que constroem um modelo explícito durante o treinamento) são drasticamente mais eficientes para fazer previsões, tornando-os as ferramentas preferidas para o mundo real. O KNN serve como um lembrete importante de que a simplicidade conceitual nem sempre se traduz em eficiência computacional.

\chapter[Regressão Linear]{Regressão Linear}

A Regressão Linear é um algoritmo de aprendizado supervisionado usado para modelar a relação entre uma variável dependente (alvo) e uma ou mais variáveis independentes (features/características). O objetivo é encontrar uma função linear (uma linha reta ou um hiperplano) que melhor capture essa relação, permitindo fazer previsões.

Existem dois tipos principais:

\begin{enumerate}
	\item \textbf{Regressão Linear Simples}: Envolve apenas uma variável independente para prever uma variável dependente:
	
	\begin{itemize}
		\item Fórmula: $y = \beta_0 + \beta_1x$.
		
		\item Visualização: Uma linha reta em um gráfico 2D.
	\end{itemize}
	
	\item \textbf{Regressão Linear Múltipla}: Envolve duas ou mais variáveis independentes para prever uma variável dependente.
	
		\begin{itemize}
		\item Fórmula: $y = \beta_0 + \beta_1x_1 + \beta_2x_2 + ... + \beta_nx_n$.
		
		\item Visualização: Um hiperplano em um espaço multidimensional.
	\end{itemize}
\end{enumerate}


\section{Para Que é Utilizado?}

A Regressão Linear é usada em qualquer cenário onde se quer quantificar a relação entre variáveis ou prever numérico continuo.

\begin{enumerate}
	\item \textbf{Previsão de Demanda}: Prever as vendas de um produto com base em preço, gastos com marketing, etc.
	
	\item \textbf{Economia}: Modelar a relação entre a taxa de juros e o investimento em um país.
	
	\item \textbf{Ciências Sociais}: Entender como fatores como educação e experiência impactam o salário de uma pessoa.
	
	\item \textbf{Bioinformática}: Prever o peso de uma pessoa com base em sua altura.
	
	\item \textbf{Séries Temporais}: (Com ressalvas) Fazer previsões baseadas em tendências lineares.
\end{enumerate}


\section{Conceito Central: Como a Linha é Encontrada?}

O objetivo do algoritmo é encontrar os coeficientes ($\beta_0, \beta_1, ..., \beta_n$) que definem a linha/hiperplano que melhor se ajusta aos dados de treinamento.

Mas o que significa "melhor se ajustar"? Significa que queremos minimizar a diferença entre os valores reais ($y_i$) e os valores previstos ($\hat{y}_i$) pelo nosso modelo. Essa diferença é chamada de erro ou resíduo.

A Regressão Linear encontra os melhores coeficentes minimizando a Soma dos Erros Quadráticos (Least Squares).


\subsection{A Fórmula Chave: Função de Custo}

A função que queremos minimizar é a Soma dos Erros Quadráticos (SSE - Sum of Squared Errors) ou mean Squared Error (MSE).

\textbf{Resíduo/Erro para um ponto}: $e_i = y_i - \hat{y}i$

\textbf{Soma dos Erros Quadráticos (SSE)}: $SSE = \sum{i=1}^{n} (y_i - \hat{y}i)^2$

\textbf{Mean Squared Error (MSE)}: $MSE = \frac{1}{n} \sum{i=1}^{n} (y_i - \hat{y}_i)^2$

O algoritmo ajusta os valores de $\beta$ para encontrar o múnimo desta função. Para a Regressão Linear Simples, existem fórmulas fechadas (fórmulas de cálculo) para encontrar os valores ótimos de $\beta_0$ (intercepto) e $\beta_1$ (coeficiente angular):

\begin{itemize}
	\item $\beta_1 = \frac{\sum_{i=1}^{n} (x_i - \bar{x})(y_i - \bar{y})}{\sum_{i=1}^{n} (x_i - \bar{x})^2}$
	
	\item $\beta_0 = \bar{y} - \beta_1\bar{x}$
\end{itemize}

Onde $\bar{x}$ e $\bar{y}$ são as médias das variáveis.

Para a Regressão Linear Múltipla, o cálculo é mais complexo e geralmente é feito usando álgebra linear: $\beta = (X^T X)^{-1} X^T y$ (onde $X$ é a matriz de features e $y$ é o vetor de valores alvo).


\section{Exemplo Prático: Prever o Preço de Casas}

Suponha que queremos prever o preço de uma cas ($y$) com base e seu tamanho em m² ($x$).

\textbf{Conjunto de Dados de Treinamento}:

\begin{table}[ht]
	\centering
	\begin{tabular}{cc}
		\hline
		\textbf{Tamanh (m²)} & \textbf{Preço R\$} \\
		\hline
		50 & 300.000 \\
		70 &  380.000 \\
		60 &  340.000 \\
		80 &  400.000 \\
		\hline
	\end{tabular}
\end{table}

\begin{enumerate}
	\item \textbf{Treinamento}: O algoritmo de Regressão Linear calcula os valores de $\beta_0$ e $\beta_1$ que minimizam a soma dos erros quadrados. Digamos que o resultado seja:
	
	$\beta_0 = 150.000$ (O preço base quando o tamanho é zero)
	
	$\beta_1 = 3.000$ (Cada m² adicional acrescenta R\$ 3.000 ao preço)
	
	\begin{itemize}
		\item Porntanto, a equação do modelo é: $Preço = 150.000 + 3.000 * Tamanho$
	\end{itemize}
	
	\item \textbf{Previsão}: Agora, queremos prever o preço de uma nova casa de 90 m².
	
	$Preço Previsto = 150.000 + 3.000 * 90$
	
	$Preço Previsto = 150.000 + 270.000 = R\$420.000$
\end{enumerate}


\section{Exemplo em Python}

\begin{lstlisting}[language=Python]
# Importando bibliotecas
import numpy as np
import matplotlib.pyplot as plt
from sklearn.linear_model import LinearRegression

# 1. Criar dados artificiais (anos de experiencia vs salario)
X = np.array([1, 2, 3, 4, 5, 6, 7, 8, 9, 10]).reshape(-1, 1)  # feature (anos de experiencia)
y = np.array([2500, 3000, 3500, 4000, 4200, 5000, 5200, 5800, 6000, 6500])  # target (salario)

# 2. Criar e treinar o modelo
modelo = LinearRegression()
modelo.fit(X, y)

# 3. Obter predicoes
y_pred = modelo.predict(X)

# 4. Coeficientes do modelo
print(f"Intercepto (B0): {modelo.intercept_:.2f}")
print(f"Coeficiente angular (B1): {modelo.coef_[0]:.2f}")

# 5. Visualizar os pontos e a reta ajustada
plt.scatter(X, y, color="blue", label="Dados reais")
plt.plot(X, y_pred, color="red", linewidth=2, label="Reta de Regressão")
plt.xlabel("Anos de Experiencia")
plt.ylabel("Salario")
plt.title("Regressao Linear - Exemplo")
plt.legend()
plt.show()
\end{lstlisting}

\textbf{O que esse código faz}:

\begin{enumerate}
	\item Cria um dataset simples (anos de experiência vs salário).
	
	\item Treina o modelo de Regressão Linear.
	
	\item Mostra o intercepto (B0) e o coeficiente angular (B1).
	
	\item Plota os pontos reais e a reta ajustada pelo model.
\end{enumerate}


\section{Pontos Positivos (Vantagens)}

\begin{itemize}
	\item \textbf{Extremamente Interpretável}: A saída do modelo é uma equação linear simples. É muito fácil entender o impacto de cada variável independente. Exemplo: "Para cada m² adicional, o preço aumenta, em média, R\$ 3,000".
	
	\item \textbf{Computacionalmente Eficiente}: Muito rápido para treinar e fazer previsões, mesmo com um grande número de features. A solução de mínimo quadrados ordinários (OLS) tem uma forma fechada.
	
	\item \textbf{Base Sólida}: Serve como um excelente ponto de partida (baseline) para problemas de regressão. Se um modelo não performar significativamente melhor que uma regressão linear, o modelo linear é provavelmente a escolha mais simples e robusta.
	
	\item \textbf{Bem Estabelecido}: É um algoritmo maduro, com fortes fundamentos estatísticos que permite inferência (testes de hipóteses sobre os coeficientes, intervalos de confiaça).
\end{itemize}


\section{Pontos Negativos (Desvantagens e Suposições)}

A Regressão Linear faz fortes suposições sobre os dados. Violar essas suposições leva a resultados não confiáveis.

\begin{itemize}
	\item \textbf{Relação Linear}: Assume que a relação entre as variáveis dependentes e independentes é linear. Ele não consegue capturar relacionamentos não-lineares complexos (ex: crescimento exponencial).
	
	\item \textbf{Multicolinearidade}: O desempenho do modelo se degrada se as variáveis independentes estiverem altamente correlacionadas entre si. Isso torna a estimativa dos coeficientes instável e de difícil interpretação.
	
	\item \textbf{Variância Constante (Homocedasticidade)}: Assume que a variância dos erros (resíduos) é constante em todos os níveis das variáveis independentes. Se a vairância dos erros não for constante (heterocedasticidade), as estimativas são ineficientes.
	
	\item \textbf{Independência dos Resíduos}: Os resíduos não devem ser correlacionados entre si (ex: em dados de séries temporais, esse é um problema comum chamado de autocorrelação).
	
	\item \textbf{Distribuição Normal dos Resíduos}: Para que os testes de hipóteses e intervalos de confiança sejam válidos, os resíduos devem ser aproximadamente normalmente distribuídos.
	
	\item \textbf{Sensibilidade a Outliers}: Como a função de custo usa quadrado dos erros, outliers têm um poso desproporcionamente alto na determinação da linha de regressão, puxando-a em sua direção.
\end{itemize}


\section{Conclusão}

A Regressão Linear é uma ferramenta fundamental e indispensável. Sua simplicidade, interpretabilidade e eficiência a tornam a primeira opção para explorar relações entre variávies e estabelecer uma baseline sólida para problemas de regressão.

No entanto, é crucial lembrar que ela é um modelo "ingênuo" no sentido de que faz suposições fortes sobre o mundo. No contexto de Big Data e ML, onde os relacionamentos são frequentemente não-lineares e os dados complexos, a Regressão Linear raramente será o modelo final e mais preciso.

Ela é frequentemente superada por algoritmos mais flexíveis como Árvore de Decisão, Random Forest e Redes Neurais. Ainda assim, entender a Regressão Linear é o primeiro passo para compreender modelos de regressão mais complexos e é uma ferramenta valiosíssima no kit de qualquer cientista de dados.

\chapter{Mineração de Dados}

A Mineração de Dados é o processo de explorar e analisar grandes volumes de dados (geralmente de múltiplas fontes) para descobrir padrões, relacionamentos, tendências e informações úteis que seriam impossíveis de encontrar manualmente.

Pense nela como a "mineração" clássica: você tem uma montanha de terra e rocha (os dados brutos) e usa ferramentas e técnicas sofisticadas para extrair pepitas de ouro puro (o conhecimento valioso).

Ela está intrinsecamente ligada ao Big Data (que fornece o material bruto em grande volume, variedade e velocidade) e ao Aprendizado de Máquina (que fornece muitas das ferramentas automatizadas para encontrar os padrões).


\section{O Processo de Mineração de Dados}

O framework mais famoso e utilizado para guiar projetos de mineração de dados é o CRISP (Cross-Industry Standard Process for Data Mining). Ele é cíclico e iterativo.

\begin{enumerate}
	\item \textbf{Compreensão do Negócio}:
	
	\begin{itemize}
		\item Definir objetivos.
		\item Perguntar: Qual problema precisa ser resolvido?
	\end{itemize}

	\item \textbf{Compreensão dos Dados}:
	
	\begin{itemize}
		\item Coleta dos dados.
		\item Exploração inicial (estatísticas, gráficos, limpeza).
	\end{itemize}
	
	\item \textbf{Preparação dos Dados}:
	
	\begin{itemize}
		\item Limpeza (remover outliers, lidar com valores faltantes).
		\item Transformar (normalizar, redução de dimensionamento, feature engineering).
	\end{itemize}
	
	\item \textbf{Modelagem}:
	
	\begin{itemize}
		\item Aplicar algoritmos de classificação, regressão, clisterização, associação, etc.
	\end{itemize}
	
	\item \textbf{Avaliação}:
	
	\begin{itemize}
		\item Verificar métricas de desempenho (acurácia, precisão, recall, F1, RMSE, etc.).
		\item Validar se o modelo responde ao problema de negócio.
	\end{itemize}
	
	\item \textbf{Implantar}:
	
	\begin{itemize}
		\item Colocar o modelo em uso prático (ex: recomendador na Netflix).
	\end{itemize}
\end{enumerate}

\begin{figure}
	\centering
	\includegraphics[width=0.5\linewidth]{"Cursos/02 - Ciência de Dados, Aprendizado de Máquina e Mineração de Dados/images/ProcessoMineracaoDeDados"}
	\caption{}
	\label{fig:processomineracaodedados}
\end{figure}

\chapter{Agrupamento Hierárquico}


O Agrupamento Hierárquico é uma técnica que busca construir uma estrutura de clusters em forma de árvore. O principal resultado visual e conceitual deste método é o dendrograma, um diagrama em árvore que ilustra a organização hierárquica dos cluesters.

Existem duas abordagens principais:

\begin{enumerate}

	\item \textbf{Algomerativa (Bottom-up ou de baixo para cima)}: Esta é a abordagem mais comum.
	\begin{itemize}
		\item \textbf{Início}: Cada ponto de dado começa como seu próprio cluster individual.
		\item \textbf{Processo}: Em cada passo, os dois clusters mais próximos (ou mais similares) são fundidos (merge).
		\item \textbf{Fim}: O processo se repete até que reste apenas um único cluster contendo todos os pontos de dados.
	\end{itemize}
	
	\item \textbf{Divisa (Top-down ou de cima para baixo)}:
	\begin{itemize}
		\item \textbf{Início}: Todos os pontos de dados começam em um único e gigantesco cluster.
		\item \textbf{Processo}: Em cada passo, um cluster é dividido (split) nos dois sub-clusters que são mais diferentes entre si.
		\item \textbf{Fim}: O processo se repete até que cada ponto de dado esteja em sue próprio cluster. É computacionalmente mais complexo e menos utilizado.
	\end{itemize}
	
\end{enumerate}


\textbf{O Dendrograma}

O dendrograma é a chave para enterder o Agrupamento Hierárquico. Ele mostra a sequência de fusões (ou divisões) e a distância em que cada uma ocorreu. A altura de cada "U" no dendrograma representa a distância entre os dois clusters que foram unidos. Ao "cortar" o dendrograma em uma determinada altura, você defini um número específico de clusters.


\section{Para Que é Utilizado?}

O Agrupamento Hierárquico é particularmente útil quando a relação hierárquica entre os dados é relevante para o problema.

\begin{itemize}
	\item \textbf{Biologia e Genética}: É massivamente utilizado para construir árvores filogenéticas (taxonomias de espécies) e para agrupar genes com padrões de expressão semelhantes.
	
	\item \textbf{Segmentação de Mercado}: Para entender não apenas os segmentos de clientes finais, mas também como eles se relacionam em subgrupos (ex: "clientes de alto valor" pode se dividir em "clientes leasi" e "novos clientes promissores").
	
	\item \textbf{Organização de Documentos}: Agrupar documentos de text em uma estrutura hierárquica de tópicos e sub-tópicos.
	
	\item \textbf{Análise de Redes Sociais}: Identificar comunidades e estrutura hierárquica dentro delas.
\end{itemize}


\section{Fórmulas (Critérios de Ligação - Linkage)}

A questão fundamental na abordagem aglomerativa é: como medimos a "distância" entre dois clusters (que pode ter múltimos pontos) para decidir quais fundir? A resposta está nos critérios de ligação (linkage criteria).

Sejam $C_i$ e $C_j$ dois clusters:

\begin{itemize}
	\item \textbf{Single Linkage (Ligação Mínima)}: A distância entre os clusters é a distância entre o par de pontos mais próximos (um de cada cluster).
	
	\begin{center}
			$d(C_i, C_j) = \frac{min}{x \in C_i, y \in C_j} d(x, y)$
	\end{center}
	
	\textit{Pode criar clusters longos e finos (efeito channing). Sensível a outliers.}
	
	\item \textbf{Complete Linkage (Ligação Máxima)}: A distância é definida pelo para de pontos mais distante.
	
	\begin{center}
			$d(C_i, C_j) = \frac{max}{x \in C_i, y \in C_j} d(x, y)$
	\end{center}
	
	\textit{Tendência a criar clusters compactos e de tamanho similar. Menos sensível a ruído.}
	
	\item \textbf{Average Linkage (Ligação Média)}: A distância média de todas as distâncias entre todos os pares de pontos (um de cada cluster).
	
	\begin{center}
			$d(C_i, C_j) = \frac{1}{|C_i||C_j|} \sum_{x \in C_i, y \in C_j}d(x, y)$
	\end{center}
	
	\textit{Um bom meio-termos entre a Single e Complete linkage}.
	
	\item \textbf{Ward's Linkage}: Um dos métodos mais populares. Ele funde os dois clusters que levam ao menor aumento na variância total intra-cluster. O objectivo é encontrar os clusters mais compactos e esféricos possíveis.
	
	\textit{Tendência a criar clusters de tamanho similar e muito compactos.}
\end{itemize}

A distância entre pontos individuais (a base de tudo) é normalmente a Distância Euclidiana, mas outras como Manhattan ou Minkowski também podem ser usadas.

\section{Exemplos}

\subsection{Exemplo Descritivo: Agrupando Cidades}

Imagine que temos 5 cidades (A, B, C, D, E) e suas distâncias.

\begin{enumerate}
	\item \textbf{Passo 1}: Cada cidade é um cluster: {A}, {B}, {C}, {D}, {E}.
	
	\item \textbf{Passo 2}: Calculamos a matriz de distância e descobrimos que A e C são as mais próximas. Fundimos elas, Nosos clusters agora são: {A, C}, {B}, {D}, {E}.
	
	\item \textbf{Passo 3}: Recalcumaos as distâncias entre os novos clusters (usando um critério de ligação, ex: avarage linkage). Agora, descobrimos que D e E são os mais próximos. Fundimos. Clusters: {A, C}, {B}, {D, E}.
	
	\item \textbf{Passo 4}: Repetimos os processo. Talvez {B} seja mais próximo do cluster {A, C}. Fundimos. Clusters: {A, C, B}, {D, E}.
	
	\item \textbf{Passo 5}: Finalmente, fundimos os dois últimos clusters. Cluster final: {A, C, B, D, E}. Observe a Figura~\ref{fig:dendrograma}.
\end{enumerate}

\begin{figure}
	\centering
	\includegraphics[width=0.8\linewidth]{"Cursos/02 - Ciência de Dados, Aprendizado de Máquina e Mineração de Dados/images/Dendrograma.png"}
	\caption{Dendrograma do Agrupamento de Cidades}
	\label{fig:dendrograma}
\end{figure}


\subsection{Exemplo em Python}

Vamos usar a biblioteca SciPy para criar o dendrograma e Scikit-learn para obter os clusters finais.

\begin{lstlisting}[language=python]

# --- Passo 0: Importar as bibliotecas necessárias ---
import pandas as pd
import numpy as np
import matplotlib.pyplot as plt
from scipy.cluster.hierarchy import dendrogram, linkage

# --- Passo 1: Criar Dados Sinteticos para as Cidades ---
# Vamos criar coordenadas (x, y) que forcem a sequencia de fusao desejada:
# 1. A e C sao muito proximas.
# 2. D e E sao as proximas mais proximas.
# 3. B e mais proximo do grupo {A,C} do que de {D,E}.
# 4. Os dois grandes grupos {A,C,B} e {D,E} sao distantes.
data = {
	'Cidade': ['A', 'B', 'C', 'D', 'E'],
	'coord_x': [2.0, 4.0, 2.5, 9.0, 9.5],
	'coord_y': [2.0, 5.0, 2.5, 8.0, 8.8]
}
df = pd.DataFrame(data)

# Extrair apenas as coordenadas para o algoritmo de clustering
X = df[['coord_x', 'coord_y']].values
labels = df['Cidade'].values

print("--- Coordenadas das Cidades ---")
print(df)
print("\n")


# --- Passo 2: Calcular o Agrupamento Hierarquico ---
# Usaremos o metodo de 'ward', que tende a encontrar clusters esfericos
# e e muito eficaz. Ele minimiza a variancia dentro dos clusters que sao fundidos.
linked = linkage(X, method='ward')


# --- Passo 3: Visualizar os Resultados ---
# Criar uma figura com dois subplots: um para o scatter e outro para o dendrograma
fig, (ax1, ax2) = plt.subplots(1, 2, figsize=(16, 7))
fig.suptitle('Exemplo de Agrupamento Hierarquico com Cidades Sinteticas', fontsize=16)

# Subplot 1: Scatter plot com a localizacao das cidades
ax1.scatter(df['coord_x'], df['coord_y'], s=100)
ax1.set_title('Localização das Cidades')
ax1.set_xlabel('Coordenada X')
ax1.set_ylabel('Coordenada Y')
ax1.grid(True)
# Adicionar rotulos para cada cidade
for i, txt in enumerate(labels):
ax1.annotate(txt, (df['coord_x'][i], df['coord_y'][i]), xytext=(5,5), textcoords='offset points', fontsize=12)

# Subplot 2: Dendrograma
ax2.set_title('Dendrograma Resultante')
ax2.set_xlabel('Cidades')
ax2.set_ylabel('Distancia (Ward)')
dendrogram(
linked,
orientation='top',
labels=labels,
distance_sort='descending',
show_leaf_counts=True
)
ax2.grid(axis='y', linestyle='--')

plt.tight_layout(rect=[0, 0.03, 1, 0.95])
plt.show()

\end{lstlisting}


\section{Pontos Positivos e Negativos}

\subsection{Pontos Positivos}

\begin{itemize}
	\item \textbf{Não Requer o Número de Clusters (K) e Priori}: Esta é uma grande vantagem sobre o K-Means. O dendrograma permite que o analista explore diferentes números de clusters e escolha o que faz mais sentido para o problema.
	
	\item \textbf{Visualização Intuitiva}: O dendrograma é uma ferramenta poderosa para visualizar a estrutura e as relações de similaridade nos dados.
	
	\item \textbf{Captura a Hierarquia}: O resultado não é uma partição "plana", mas uma estrutura hierárquica que pode ter um significado real no domímio do problema (ex: taxonimia biológica).
	
	\item \textbf{Flexibilidade}: Pode funcionar com qualquer métrica de distância e critério de ligação.
\end{itemize}


\subsection{Pontos Negativos}

\begin{itemize}
	\item \textbf{Alto Custo Computacional e de Memória}: A complexidade de tempo é tipicamente $O (n^3)$ e a de memória $O (n^2)$ (para armazenar a matriz de distância). Isso torna o algortimo inviável para datasets grandes (Big Data).
	
	\item \textbf{Decisões Irreversíveis}: Uma vez que uma fusão (ou divisão) é feita, ela não pode ser desfeita. Uma fusão inicial ruim pode impactar toda a estrutura da árvore.
	
	\item \textbf{Sensibilidade aos Parâmetros}: O resultado pode mudar drasticamente dependendo da mátrica de distância e do critério de ligação escolhidos.
	
	\item \textbf{Dificuldade de Interpretação em Larga Escala}: Para datasets com milhares de pontos, o dendrograma se torna uma imagem densa e impossível de interpretar visualmente.
\end{itemize}


\section{Conclusão}

O Agrupamento Hierárquico é uma ferramenta poderosa para exploração e análise de dados em pequena e média escala. Sua capacidade de revelar estruturas hierárquicas naturais e a não necessidade de definir K previamente o tornam inestimável para análise inicial de dados.

No entanto, no mundo do Big Data, sua ineficiência computacional o confina a subconjuntos de dados ou a casos de uso específicos onde a hierarquia é essencial. Para conjuntos de dados muito grandes, algoritmos como K-Means (com inicialização inteligente como K-Means++) ou DBSCAN (ótimo para dados com ruído e formatos arbitrários) são geralmente escolhas mais práticas e escaláveis. O agrupamento hierárquico é a lente de aumento perfeita para uma análise detalhada, mas não é o trator para arar grandes campos de dados.

\chapter{Agrupamento Particional}

O Agrupamento Particionado é um método de aprendizado não supervisionado que divide um conjunto de dados em um número pré-definido, $k$, de grupos mutuamente exclusivos, chamados de clusters. O objetivo é organizar os dados de modo que:

\begin{itemize}
	\item \textbf{Dentro de um cluster (Coesão)}: os objetos sejam o mais similares possível.
	
	\item \textbf{Entre clusters (Separação)}:  os objetos de clusters diferentes sejam o mais dissimilares possível.
\end{itemize}

A palavra "particionado" vem do fato de que cada ponto de dados pertence a exatamente um cluster (partição rígida).


\section{Para Que é Utilizado?}

O K-Means é extremamente versátil e aplicado em diversas áreas, graças a sua eficiência e simplicidade.

\begin{itemize}
	\item \textbf{Segmentação de Clientes}: Agrupar clientes com base em seu comportamento de compra, dados demográficos ou histórico de navegação para criar campanhas de marketing direcionadas.
	
	\item \textbf{Organização de Documentos}: Agrupar notícias ou artigos por tópicos similares.
	
	\item \textbf{Compressão de Imagens}: Reduzir o número de cores em uma umagem agrupando cores similares e substituindo-as pela cor do centroide do cluster.
	
	\item \textbf{Detecção de Anomalias}: Pontos de dados que ficam muito distantes de qualquer centroide após o agrupamento podem ser considerados anomalias ou outliers.
	
	\item \textbf{Análise de Dados Genômicos}: Agrupar genes com padrões de expressão semelhantes.
\end{itemize}

\section{O Algoritmo K-Means}

O K-Means é o algoritmo mais famoso de agrupamento particionado. Sua ideia é simples: encontrar $k$ centróides (o "centro" de cada cluster) e atribuir cada ponto ao centróide mais próximo, minimizando a variância dentro dos clusters.

\textbf{Fórmula}

O algoritmo busca minimizar a Soma dos Erros Quadráticos (SSE - Sum of Squared Errors), também conhecida como inércia dentro do cluster.

\begin{center}
	$E = \sum_{i=1}^{k}\sum_{x \in C_i}^{} ||x, \mu_i||^2$
\end{center}

\begin{itemize}
	\item $k$: Número de clusters.
	
	\item $C_i$: Conjunto de pontos no cluster $i$.
	
	\item $\mu_i$: Centroid do cluster $i$.
	
	\item $||x, \mu_i||^2$: Distância Euclidiana ao quadrado entre o ponto $x$ e o seu centroide.
\end{itemize}

\textbf{Passos do Algoritmo (Explicado com a Figura Acima)}

\begin{enumerate}
	\item \textbf{Inicialização}: Escolha aleatória de $k$ centróides iniciais (\ref{fig:k-means-inicializacao}).
	
	\begin{figure}
		\centering
		\includegraphics[width=0.7\linewidth]{"Cursos/02 - Ciência de Dados, Aprendizado de Máquina e Mineração de Dados/images/K-Means-Inicializacao"}
		\caption{Inicialização}
		\label{fig:k-means-inicializacao}
	\end{figure}
	
	\item \textbf{Atribuição de Pontos}: Cada ponto é atribuído ao centróide mais próximo, formando $k$ clusters temporários (\ref{fig:k-means-atribuicao}).
	
	\begin{figure}
		\centering
		\includegraphics[width=0.7\linewidth]{"Cursos/02 - Ciência de Dados, Aprendizado de Máquina e Mineração de Dados/images/K-Means-Atribuicao"}
		\caption{Atribuição}
		\label{fig:k-means-atribuicao}
	\end{figure}
	
	\item  \textbf{Atualização dos Centroides}: O centróide de cada cluster é recalculado como a média de todos os pontos naquele cluster (\ref{fig:k-means-atualizacao}).
	
	\begin{figure}
		\centering
		\includegraphics[width=0.7\linewidth]{"Cursos/02 - Ciência de Dados, Aprendizado de Máquina e Mineração de Dados/images/K-Means-Atualizacao"}
		\caption{Atualização}
		\label{fig:k-means-atualizacao}
	\end{figure}
	
	\item \textbf{Iteração}: Os passos 2 e 3 são repetidos até que não haja mais mudanças na atribuição dos pontos ou um número máximo de iterações seja atingido (convergência para o Resultado Final).
\end{enumerate}


\section{Exemplos}

\subsection{Exemplo Descritivo: Segmentação de Clientes de E-commerce}

Imagine um e-commerce que quer segmentar seus clientes com base em duas variáveis: "Frequência de Visitas" e "Valor Gasto".

\begin{enumerate}
	\item \textbf{Escolha do k}: A equipe de marketing decide que 3 segmentos (ex: "ocasionais", "fiéis" e "super fãs") é um bom ponto de partida.

	\item \textbf{Inicialização}: O algoritmo K-Means escolhe 3 clientes aleatórios como os centroides iniciais.
	
	\item \textbf{Execução}:
	
	\begin{itemize}
			\item Na Primeira iteração, cada cliente é atribuído ao centróide mais próximo.
			
			\item Nas próximas iterações, os clientes são realocados conforme os centróides se movem.
			
			\item O processo converge quando os centróides se estabilizam e os clusters estão bem definidos.
	\end{itemize}
		
	\item \textbf{Resultado}:
	
	\begin{itemize}
		\item \textbf{Cluster 1 (Fiéis)}: Alta frequência, alto valor gasto.
		
		\item \textbf{Cluster 2 (Ocasionais)}: Média frequência, médio valor gasto.
		
		\item \textbf{Cluster 3 (Em Risco)}: Baixa frequência, baixo valor gasto (ou que não compram há tempos).
	\end{itemize}
\end{enumerate}

A empresa pode agora criar campanhas de marketing específicas para cada grupo.

\subsection{Exemplo Prático em Python}

Vamos usar scikit-learn para aplicar o K-Means em dados sintéticos com 4 clusters bem definidos.

\begin{lstlisting}[language=Python]
# Importando bibliotecas necessarias
from sklearn.cluster import KMeans
from sklearn.datasets import make_blobs
import matplotlib.pyplot as plt
import numpy as np

# 1. Criar um conjunto de dados sintetico para demonstracao
X, y_true = make_blobs(n_samples=300, centers=4, cluster_std=0.60, random_state=0)

# 2. Visualizar os dados originais
plt.scatter(X[:, 0], X[:, 1], s=50)
plt.title("Dados Originais")
plt.show()

# 3. Instanciar e executar o algoritmo K-Means com k=4
kmeans = KMeans(n_clusters=4, n_init=10, random_state=0)
kmeans.fit(X)
y_kmeans = kmeans.predict(X)

# 4. Visualizar os resultados do clustering
# Pontos coloridos por cluster
plt.scatter(X[:, 0], X[:, 1], c=y_kmeans, s=50, cmap='viridis')
# Plotar os centroides finais
centers = kmeans.cluster_centers_
plt.scatter(centers[:, 0], centers[:, 1], c='red', s=200, alpha=0.75, marker='X')
plt.title("Resultado do Clustering K-Means")
plt.show()

# 5. Mostrar as coordenadas dos centroides e a inertia (SSE)
print("Coordenadas dos Centroides:")
print(centers)
print(f"\nInércia (SSE): {kmeans.inertia_:.2f}")
\end{lstlisting}

\textbf{Saída Experada}:

\begin{itemize}
	\item Um primeiro gráfico com pontos azuis espalhados (\ref{fig:k-means-grafico-1}).
	
	\begin{figure}
		\centering
		\includegraphics[width=0.7\linewidth]{"Cursos/02 - Ciência de Dados, Aprendizado de Máquina e Mineração de Dados/images/K-Means-Grafico-1.png"}
		\caption{Atualização}
		\label{fig:k-means-grafico-1}
	\end{figure}
	
	\item segundo gráfico com os pontos coloridos em 4 grupos diferentes e um "X" vermelho marcando o centro de cada grupo (\ref{fig:k-means-grafico-2}).
	
	\begin{figure}
		\centering
		\includegraphics[width=0.7\linewidth]{"Cursos/02 - Ciência de Dados, Aprendizado de Máquina e Mineração de Dados/images/K-Means-Grafico-2.png"}
		\caption{Atualização}
		\label{fig:k-means-grafico-2}
	\end{figure}
	
	\item As coordenadas dos centróides e o valor da SSE serão impressos no console.
\end{itemize}

\section{Pontos Positivos e Negativos}

\subsection{Pontos Positivos}

\begin{itemize}
	\item \textbf{Simplicidade e Interpretabilidade}: O algoritmo é fácil de entender e implementar, e os resultados (clusters definidos por centroides) são fáceis de interpretar.
	
	\item \textbf{Eficiência Computacional}: O K-Means é relativamente rápido e escala bem para grandes conjunto de dados, o que torna desejável para Mineração de Dados.
	
	\item \textbf{Versatilidade}: Adapta-se bem a uma grande variedade de problemas e domínios.
	
	\item \textbf{Convergência Rápida}: Geralmente converge em poucas iterações.
\end{itemize}

\subsection{Pontos Negativos}

\begin{itemize}
	\item  \textbf{Necessidade de Definir $k$}: É preciso saber o número de cluster de antemão, o que nem sempre é óbvio.
	
	\item  \textbf{Sensibilidade a Inicialização}: A inicialização aleatória dos centróides pode levar a resultados diferentes (clusters sub-ótimos) a cada execução. A solução comum é rodar o algoritmo várias vezes com diferentes inicializações e escolher o melhor resultado.
		
	\item  \textbf{Sensibilidade a Outliers}: Como os centroides são baseados na média, eles são fortemente influenciados por outliers.
			
	\item  \textbf{Assume Clusters Esféricos}: O K-Means funciona melhor quando os clusters são de formato globular (esférico), de tamanhos e densidades similares. Ele tem dificuldade em encontrar clusters com formatos irregulares ou alongados.
\end{itemize}


\section{Conclusão}

O K-Means é um algoritmo fundamental no toolkit de qualquer cientista de dados. Sua simplicidade e eficiência o tornam uma excelente primeira escolha para tarefas de clustering, especialmente com grandes volumes de dados (Bid Data), e uma base sólida para comparação com algoritmos mais complexos.

No entanto, é crucial estar ciente de suas limitações. Para problemas com clusters de formato não-globular, algoritmos baseados em densidade como o DBSCAN ou baseados em hieraquia como Agrupamento Hierárquico são geralmente escolhas mais adequadas.

\chapter{Validação de Agrupamento}

A validação de agrupamento é uma etapa crítica em aprendizado não supervisionado. O objetivo é determinar o quão "bom" é um agrupamento, baseando-se em dois critérios fundamentais:

\begin{itemize}
	\item \textbf{Coesão (ou Compactação)}: Mede o quão próximos e similares são os objetos dentro do mesmo cluster. Um bom cluster é altamente coeso.
	
	\item \textbf{Separação}: Mede o quão distintos e distantes são os objetos de diferentes clusters. Bons agrupamentos têm alta separação entre seus clusters.
\end{itemize}

Como não há uma "resposta correta" pré-definida, utilizamos métricas que avaliam a estrutura interna dos dados. Essas métrias se dividem em três categorias, conforme ilustrado na figura \ref{fig:indice-validacao}) abaixo.

\begin{figure}[H]
	\centering
	\includegraphics[width=0.7\linewidth]{"Cursos/02 - Ciência de Dados, Aprendizado de Máquina e Mineração de Dados/images/Indice-Validacao.png"}
	\caption{Principais Índices}
	\label{fig:indice-validacao}
\end{figure}


\section{Para que é Utilizado?}

A validação é utilizada para responder a três perguntas cruciais:

\begin{enumerate}
	\item \textbf{Determinar o número ideal de clusters ($k$)}: Esta é a aplicação mais comum. A validação nos ajuda a encontrar o valor de $k$ que melhor representa a estrutura natural dos dados.
	
	\item \textbf{Comparar diferentes algoritmos ou configurações}: Qual algoritmo (K-Means, Hierárquico) ou qual configuração (ex: tipo de linkage no Hierárquico) funciona melhor para os meus dados?	
	
	\item \textbf{Avaliar a qualidade e o significado}: O agrupamento encontrado é estatisticamente significativo ou é apenas um resultado aleatório?
\end{enumerate}

\section{Inspeção Visual}

Consiste em visualizar a matriz de dissimilaridades (ou similaridades) dos dados, mas ordenando os objetos de acordo com os clusters encontrados pelo algoritmo.

É uma ferramenta de exploração inicial rápida. Em uma matriz termo-documento, por exemplo, os clusters bem formados aparecem como "blocos" ou "quadrados" algongados na diagonal principal, indicando que objestos do mesmo cluster são similares entre si e dissimilares dos de outros clusters.

\begin{itemize}
	\item \textbf{Pontos Positivos}: Intuitiva, rápida e boa para uma primeira impressão.
	
	\item \textbf{Pontos Negativos}: Extremamente subjetiva. Torna-se inviável em conjuntos de dados com milhares de observações. Não deve ser usada como único método de validação.
\end{itemize}

A figura \ref{fig:matriz-similaridade-agrupadas} mostra um exemplo de Matriz de Similaridades onde a estrutura de cluster esta bem definida.

\begin{figure}[H]
	\centering
	\includegraphics[width=0.7\linewidth]{"Cursos/02 - Ciência de Dados, Aprendizado de Máquina e Mineração de Dados/images/Matriz-Similaridades-Agrupadas.png"}
	\caption{Matriz de Similaridades com clusters agrupados}
	\label{fig:matriz-similaridade-agrupadas}
\end{figure}

Ja a figura \ref{fig:matriz-similaridade-dispersas}, observe que dados sem uma estrutura de cluster bem definida se destacam menos na inspeção visual.

\begin{figure}[H]
	\centering
	\includegraphics[width=0.7\linewidth]{"Cursos/02 - Ciência de Dados, Aprendizado de Máquina e Mineração de Dados/images/Matriz-Similaridades-Dispersas.png"}
	\caption{Matriz de Similaridades com clusters dispersos}
	\label{fig:matriz-similaridade-dispersas}
\end{figure}

\subsection{Exemplo em Python}

\begin{lstlisting}[language=python]
import numpy as np
import matplotlib.pyplot as plt
from scipy.spatial.distance import pdist, squareform
from sklearn.datasets import make_blobs
from sklearn.cluster import KMeans

# Gerar dados sinteticos
X, y = make_blobs(n_samples=100, centers=3, random_state=42)

# Calcular matriz de distancias
dist_matrix = squareform(pdist(X))

# Fazer clustering
kmeans = KMeans(n_clusters=3, random_state=42).fit(X)
labels = kmeans.labels_

# Ordenar a matriz de distancias pelos labels
sort_idx = np.argsort(labels)
sorted_matrix = dist_matrix[sort_idx][:, sort_idx]

# Plotar
plt.figure(figsize=(10, 8))
plt.imshow(sorted_matrix, cmap='viridis', interpolation='nearest')
plt.colorbar()
plt.title('Matriz de Dissimilaridades Ordenada por Cluster')
plt.show()
\end{lstlisting}

\section{Índices de Validade Interna}

Avaliam a qualidade usando apenas os próprios dados e a atribuição de clusters, sem informações extermas.

\subsection{Erro Quadrático (SSE - Sum of Squared Errors)}

É a função custo do algoritimo K-Means . Mede a coesão interna (quão compactos são os clusters).

\begin{itemize}
	\item \textbf{Fórmula}:
	
	\begin{itemize}
			\item $SSE = \sum_{i=1}^{k} \sum_{\mathbf{x} \in C_i} ||\mathbf{x} - \mathbf{\mu}_i||^2 $
			
			\item $k$: Número de clusters.
			
			\item $C_i$: Conjunto de pontos noncluster $i$.
			
			\item $\mu_i$: Centróide (vetor de médias) do cluster $i$.
	\end{itemize}
	
	
	\item \textbf{Para que serve}: Escolher a melhor execução de um algoritmo após várias inicializações (evitando um resultado ruim por azar na escolha inicial dos centróides). É base para o Método do Cotovelo (Elbow Method) para encontrar o número ideal de clusters $k$.
	
	\item \textbf{Interpretação}: Quanto maior o valor, mais compactos e coesos são os clusters.
	
	\item \textbf{Pontos Negativos}: Sempre diminui à medida que $k$ aumenta. Se $k=n$, $SSE = 0$. Portant, não pode ser usado sozinho para escolher $k$.
	
	\item \textbf{Interpretação}: Um SSE de 150.4 indica que a distância total dos pontos para seus centróides é menor que um SSE de 320.7, sugerindo clusters potencialmente mais compactos no primeiro caso.
	
	\item \textbf{Exemplo em Python (Elbow Method)}:
	
	\begin{lstlisting}[language=python]
	from sklearn.cluster import KMeans
	from sklearn.datasets import make_blobs
	
	# Gerar dados
	X, y = make_blobs(n_samples=100, centers=3, random_state=42)
	
	# Executar K-Means
	kmeans = KMeans(n_clusters=3, random_state=42, n_init='auto').fit(X)
	
	# Obter o SSE (chamado de 'inertia_' no scikit-learn)
	sse = kmeans.inertia_
	print(f"SSE para k=3: {sse:.4f}")
	# Output: SSE para k=3: ~300.0 (valor variavel)
	\end{lstlisting}
\end{itemize}


\subsection{Correlação Cofenética}

Avalia a qualidade de um agrupamento hierárquico (ex: ligação simples, completa, média). Mede quão bem o dendrograma preserva as distâncias originais entre os pares de pontos.

\begin{itemize}
	\item \textbf{Para que server}: Validar se o dendrograma gerado é uma representação fiel das distâncias originais entre os dados.
	
	\item \textbf{Como funciona}:
	
	\begin{enumerate}
		\item Calcula a Matriz de Distâncias Originais (ex: Euclidiana) entre todos os pares de pontos.
		
		\item Gera a Matriz Cofenética a partir do dendrograma. A distância cofenética entre dois pontos é a altura no dendrograma onde eles são unidos pela primeira vez.
		
		\item Calcula a Correlação de Pearson entre os valores das duas matrizes (considerando apenas a triangular inferior ou superior, pois são simétricas).
	\end{enumerate}
	
	\item \textbf{Pontos Positivos}: Única métrica interna robusta projetada especificamente para avaliar a qualidade de dendrogramas.
	
	\item \textbf{Pontos Negativos}: Aplicável apenas para agrupamentos hierárquicos. Cálculo da matriz de distâncias é $O(n_2)$, tornando-se inviável para conjuntos de dados muito grandes.
		
	\item \textbf{Interpretação}: O valor da correlação varia entre -1 e 1. Quanto mais próximo de 1, melhor a qualidade do dendrograma e do agrupamento hierárquicos.
	
	\item \textbf{Exemplo em Python}:
	
	\begin{lstlisting}[language=python]
	from scipy.cluster.hierarchy import linkage, cophenet
	from scipy.spatial.distance import pdist
	import numpy as np
	
	# Gerar os mesmos dados
	X, y = make_blobs(n_samples=100, centers=3, random_state=42)
	
	# Calcular a matriz de distancias condensada
	distancias = pdist(X)
	
	# Realizar agrupamento hierarquico
	Z = linkage(distancias, method='average')
	
	# Calcular a correlação cofenetica
	c, coph_dists = cophenet(Z, distancias)
	print(f"Correlação Cofenética: {c:.4f}")
	# Output: Correlação Cofenética: ~0.87 (valor variável)
	\end{lstlisting}
\end{itemize}


\section{Índices de Validade Relativa}

Usados para comparar diferentes agrupamentos (ex: diferentes algoritmos ou diferentes valores de $k$).


\subsection{Índice de Silhueta (Silhouette Score)}

É o índice mais popular e intuitivo. Ele calcula, para cada ponto, o quão bem ele está em seu próprio cluster comparado a outros clusters.

\begin{itemize}
	\item \textbf{Para que serve}: Escolher o número ideal de clusters $k$ e comparar diferentes algoritmos.
	
	\textbf{Fórmula}: Para cada amostra $i$:
	
	\begin{enumerate}
		\item \textbf{Calcule $a(i)$}: Distância média entre $i$ e todos os outros pontos no mesmo cluster (Coesão).
		
		\item \textbf{Calcule $b(i)$}: Distância média entre $i$ e todos os pontos no cluster mais próximo do qual $i$ faz parte (Separação).
		
		\item Calcule o score da silhueta para a amostra $i$
		
		\begin{center}
			$s(i) = \frac{b(i) - a(i)}{max{a(i), b(i)}}$
		\end{center}
		
	\end{enumerate}
	
	\item \textbf{Interpretação}:
	
	\begin{itemize}
		\item $s(i) \approx 1$: O ponto está bem posicionado em seu cluster.
		
		\item $s(i) \approx 0$:  O ponto está muito próximo da fronteira entre dois clusters.
		
		\item $s(i) \approx -1$: O ponto provavelmente foi atribuído ao cluster errado.
		
		\item Um score de silhueta médio de 0.65 para k=3 sugere uma estrutura de clusters boa e muito melhor do que um score de 0.12 para k=2, indicando que k=3 é a melhor escolha.
	\end{itemize}
	
	\item \textbf{Exemplo em Python}:
	
	\begin{lstlisting}[language=python]
	from sklearn.metrics import silhouette_score
	
	# Usando os labels do K-Means anterior
	silhouette_avg = silhouette_score(X, kmeans.labels_)
	print(f"Silhouette Score para k=3: {silhouette_avg:.4f}")
	# Output: Silhouette Score para k=3: ~0.55 (valor variável)
	
	# E comum calcular para vários k
	for k in [2, 3, 4, 5]:
		kmeans = KMeans(n_clusters=k, random_state=42, n_init='auto')
		kneans.fit(X)
		score = silhouette_score(X, kmeans.labels_)
		print(f"k = {k}: Silhouette Score = {score:.4f}")
	\end{lstlisting}
\end{itemize}


\section{Índices de Validação Externa}

Comparam a partição encontrada pelo algoritmo com uma partição de referencia ("ground truth") que é considerada correta.

\subsection{Índice RAND (RAND Index)}

Mede a similaridade entre duas partições de dados: a encontrada pelo algoritmo ($U$) e uma partição de referência ($V$). Calcula a proporção de pares de pontos que forma concordantemente agrupados em $U$ e $V$.

\begin{itemize}
	\item \textbf{Para que server}: Validar os resultados de um algoritmo de clustering quando se possui labels verdadeiros. Comum em pesquisa para comparar algoritmos usando bases de dados conhecidas.
	
	\item \textbf{Formula}:
	
	\begin{center}
		$[ RI = \frac{a + b}{a + b + c + d} = \frac{a + b}{\binom{n}{2}} ]$
	\end{center}
	
	\begin{itemize}
		\item $a$: Pares no mesmo cluster em $U$ e no mesmo cluster em $V$.
		
		\item $b$: Pares em clusters diferentes em $U$ e em clusters diferentes em $V$.
		
		\item $c$: Pares no mesmo cluster em $U$ mas em clusters diferentes em $V$.
		
		\item $d$: Pares em clusters diferentes em $U$ mas no mesmo cluster em $V$.
		
		\item $n$: Número total de pontos.
	\end{itemize}
	
	\item \textbf{Pontos Positivos}: Fácil de interpretar. Leva em conta todos os pares de pontos.
	
	\item \textbf{Pontos Negativos}: Crucialmente, requer uma rotulação prévia, o que raramente está disponível em problemas reais de clusterint puro. O valor esperado para duas partições aleatórias não é zero, o que dificulta a interpretação (essa limitação é corrigida pelo Adjusted Rand Index).
	
	\item \textbf{Interpretação}: Varia de 0 a 1. Quanto mais próximo de 1, maior a concordância entre o clustering e a partição de referência. Se o RI for 0.88, significa que 88\% dos pares de pontos tiveram o mesmo destino (ficarem juntos ou separados) tanto no clustering quanto na classificação verdadeira.
	
	\item \textbf{Exemplo em Python}:
	
	\begin{lstlisting}[language=python]
	from sklearn.metrics import rand_score
	
	# Usando os labels verdadeiros (y) e os labels do K-Means (kmeans.labels_)
	ri = rand_score(y, kmeans.labels_)
	print(f"RAND Index: {ri:.4f}")
	# Output: RAND Index: ~0.95 (valor variável, mas alto pois os dados são bem separados)
	\end{lstlisting}
	
\end{itemize}

\section{Exemplos}

\subsection{Exemplo Descritivo: Validando Segmentos de Clientes}

Uma empresa rodou um K-Means em seus dados de clientes e obteve 3 clusters. Para validar, o analista:

\begin{enumerate}
	\item \textbf{Validação Interna/Relativa}: Ele calcula o Score de Silhueta médio para o resultado com $k=3$ e obtém 0.85. Ele também testa com $k=2$ (score 0.70) e $k=4$ (score 0.65). Ele conclui que $k=3$ é a melhor escolha, pois maximizou a silhueta.
	
	\item \textbf{Validação Externa (se possível)}: A empresa tinha uma classificação manual antiga para 100 de seus clientes VIPs. O analista usa essa classificação como "gabarito" e calcula o RAND Index entre o seu agrupamento e o gabarito. Ele obtém um score de 0.92, indicando que os clusters encontrados pelo algoritmo correspondem muito bem à segmentação manual, validando a relevância do modelo para o negócio.
\end{enumerate}


\subsection{Exemplo em Python}

Este código completo demonstra como aplicar os índices WCSS (Cotovelo), Silhueta e RAND para validar um agrupamento.

\begin{lstlisting}[language=python]
# --- Passo 0: Importar as bibliotecas necessarias ---
import matplotlib.pyplot as plt
import numpy as np
from sklearn.datasets import make_blobs
from sklearn.cluster import KMeans
from sklearn.metrics import silhouette_score, rand_score
from scipy.cluster.hierarchy import dendrogram, linkage, cophenet
from scipy.spatial.distance import pdist

# --- Passo 1: Gerar Dados Sinteticos ---
# Criar dados com 4 centros bem definidos. O objetivo e que a validacao aponte k=4.
# X sao as coordenadas, y_true sao os rotulos verdadeiros (nosso "gabarito")
X, y_true = make_blobs(n_samples=500, centers=4, cluster_std=0.8, random_state=42)

# --- Passo 2: Validar K-Means com Indices Internos e Relativos ---
k_range = range(2, 11)
wcss = []
silhouette_scores = []

for k in k_range:
	kmeans = KMeans(n_clusters=k, random_state=42, n_init=10)
	kmeans.fit(X)
	wcss.append(kmeans.inertia_)
	silhouette_scores.append(silhouette_score(X, kmeans.labels_))

# --- Plotar os graficos de validacaoo para K-Means ---
fig, (ax1, ax2) = plt.subplots(1, 2, figsize=(16, 6))
fig.suptitle('Validacao de K-Means para Encontrar o Numero Ideal de Clusters (k)', fontsize=16)

# Subplot 1: Metodo do Cotovelo (WCSS)
ax1.plot(k_range, wcss, 'bo-')
ax1.set_xlabel('Numero de Clusters (k)')
ax1.set_ylabel('WCSS (Erro Quadratico)')
ax1.set_title('Metodo do Cotovelo')
ax1.grid(True)

# Subplot 2: Score de Silhueta
ax2.plot(k_range, silhouette_scores, 'ro-')
ax2.set_xlabel('Numero de Clusters (k)')
ax2.set_ylabel('Score de Silhueta Medio')
ax2.set_title('Analise de Silhueta')
ax2.grid(True)
plt.show()

# --- Conclusao para K-Means ---
k_ideal_silhueta = k_range[np.argmax(silhouette_scores)]
print(f"O pico no grafico de Silhueta foi em k = {k_ideal_silhueta}, com um score de {max(silhouette_scores):.2f}.")
print("O 'cotovelo' no grafico WCSS tambem sugere k=4.")
print("-" * 50)

# --- Passo 3: Validar com Indice Externo (RAND Index) ---
# Usar o k ideal encontrado (k=4) e comparar com o gabarito (y_true)
kmeans_final = KMeans(n_clusters=k_ideal_silhueta, random_state=42, n_init=10)
kmeans_final.fit(X)
rand_index = rand_score(y_true, kmeans_final.labels_)
print(f"Usando k={k_ideal_silhueta}, o RAND Index em comparacao com o gabarito e: {rand_index:.3f}")
print("Um valor proximo de 1.0 indica uma excelente correspondencia com os clusters verdadeiros.")
print("-" * 50)


# --- Passo 4: Validar Agrupamento Hierarquico com Correlacao Cofenetica ---
print("Avaliando Agrupamento Hierarquico com Correlacao Cofenetica...")
# Calcular o linkage (estrutura do dendrograma)
linked = linkage(X, method='ward')

# Calcular a correlacao cofenetica
# pdist calcula a matriz de distancia original
coph_corr, coph_dist = cophenet(linked, pdist(X))
print(f"A Correlacao Cofenetica do dendrograma e: {coph_corr:.3f}")
print("Um valor proximo de 1.0 indica que o dendrograma preserva bem as distancias originais.")

# Plotar o dendrograma para visualizacao
plt.figure(figsize=(12, 7))
plt.title('Dendrograma do Agrupamento Hierarquico')
dendrogram(linked, truncate_mode='lastp', p=12, leaf_rotation=45., show_contracted=True)
plt.xlabel('Tamanho do Cluster')
plt.ylabel('Distancia (Ward)')
plt.grid(axis='y', linestyle='--')
plt.show()

\end{lstlisting}


\section{Pontos Positivos e Negativos}

\subsection{Pontos Positivos}

\begin{itemize}
	\item \textbf{Objetividade}: Substitui a análise puramente visual por métricas quantitativas, reduzindo a subjetividade.
	
	\item \textbf{Confiança}: Aumenta a confiança de que a estrutura encontrada no agrupamento é significativa e não aleatória.
	
	\item \textbf{Orientação}: Fornece uma base sólida para decisões importantes, como a definição do número de clusters.
\end{itemize}

\subsection{Pontos Negativos}

\begin{itemize}
	\item \textbf{Custo Computacional}: Rodar o algoritmo de agrupamento múltiplas vezes para diferentes valores de $k$ pode ser demorado.
	
	\item \textbf{Ambiguidade}: Diferentes índices podem, às vezes, sugerir diferentes números ideais de $k$, exigindo uma análise mais aprofundada do contexto do problema.
	
	\item \textbf{Não Garante Relevância de Negócio}: Um agrupamento matematicamente "perfeito" pode não ser útil ou acionável do ponto de vista do negócio. A validação quantitativa deve sempre ser complementada por uma análise qualitativa dos clusters encontrados.
\end{itemize}
% ---


\end{document}
